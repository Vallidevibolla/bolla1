\documentclass[journal,12pt,twocolumn]{IEEEtran}

\usepackage{setspace}
\usepackage{gensymb}

\singlespacing


\usepackage[cmex10]{amsmath}

\usepackage{amsthm}

\usepackage{mathrsfs}
\usepackage{txfonts}
\usepackage{stfloats}
\usepackage{bm}
\usepackage{cite}
\usepackage{cases}
\usepackage{subfig}

\usepackage{longtable}
\usepackage{multirow}

\usepackage{enumitem}
\usepackage{mathtools}
\usepackage{steinmetz}
\usepackage{tikz}
\usepackage{circuitikz}
\usepackage{verbatim}
\usepackage{tfrupee}
\usepackage[breaklinks=true]{hyperref}
\usepackage{graphicx}
\usepackage{tkz-euclide}
\usepackage{float}

\usetikzlibrary{calc,math}
\usepackage{listings}
\usepackage{color} %%
\usepackage{array} %%
\usepackage{longtable} %%
\usepackage{calc} %%
\usepackage{multirow} %%
\usepackage{hhline} %%
\usepackage{ifthen} %%
\usepackage{lscape}
\usepackage{multicol}
\usepackage{chngcntr}

\DeclareMathOperator*{\Res}{Res}

\renewcommand\thesection{\arabic{section}}
\renewcommand\thesubsection{\thesection.\arabic{subsection}}
\renewcommand\thesubsubsection{\thesubsection.\arabic{subsubsection}}

\renewcommand\thesectiondis{\arabic{section}}
\renewcommand\thesubsectiondis{\thesectiondis.\arabic{subsection}}
\renewcommand\thesubsubsectiondis{\thesubsectiondis.\arabic{subsubsection}}


\hyphenation{op-tical net-works semi-conduc-tor}
\def\inputGnumericTable{} %%

\lstset{
%language=C,
frame=single,
breaklines=true,
columns=fullflexible
}
\begin{document}


\newtheorem{theorem}{Theorem}[section]
\newtheorem{problem}{Problem}
\newtheorem{proposition}{Proposition}[section]
\newtheorem{lemma}{Lemma}[section]
\newtheorem{corollary}[theorem]{Corollary}
\newtheorem{example}{Example}[section]
\newtheorem{definition}[problem]{Definition}

\newcommand{\BEQA}{\begin{eqnarray}}
\newcommand{\EEQA}{\end{eqnarray}}
\newcommand{\define}{\stackrel{\triangle}{=}}
\bibliographystyle{IEEEtran}
\providecommand{\mbf}{\mathbf}
\providecommand{\pr}[1]{\ensuremath{\Pr\left(#1\right)}}
\providecommand{\qfunc}[1]{\ensuremath{Q\left(#1\right)}}
\providecommand{\sbrak}[1]{\ensuremath{{}\left[#1\right]}}
\providecommand{\lsbrak}[1]{\ensuremath{{}\left[#1\right.}}
\providecommand{\rsbrak}[1]{\ensuremath{{}\left.#1\right]}}
\providecommand{\brak}[1]{\ensuremath{\left(#1\right)}}
\providecommand{\lbrak}[1]{\ensuremath{\left(#1\right.}}
\providecommand{\rbrak}[1]{\ensuremath{\left.#1\right)}}
\providecommand{\cbrak}[1]{\ensuremath{\left\{#1\right\}}}
\providecommand{\lcbrak}[1]{\ensuremath{\left\{#1\right.}}
\providecommand{\rcbrak}[1]{\ensuremath{\left.#1\right\}}}
\theoremstyle{remark}
\newtheorem{rem}{Remark}
\newcommand{\sgn}{\mathop{\mathrm{sgn}}}
\providecommand{\abs}[1]{\left\vert#1\right\vert}
\providecommand{\res}[1]{\Res\displaylimits_{#1}}
\providecommand{\norm}[1]{\left\lVert#1\right\rVert}
%\providecommand{\norm}[1]{\lVert#1\rVert}
\providecommand{\mtx}[1]{\mathbf{#1}}
\providecommand{\mean}[1]{E\left[ #1 \right]}
\providecommand{\fourier}{\overset{\mathcal{F}}{ \rightleftharpoons}}
%\providecommand{\hilbert}{\overset{\mathcal{H}}{ \rightleftharpoons}}
\providecommand{\system}{\overset{\mathcal{H}}{ \longleftrightarrow}}
%\newcommand{\solution}[2]{\textbf{Solution:}{#1}}
\newcommand{\solution}{\noindent \textbf{Solution: }}
\newcommand{\cosec}{\,\text{cosec}\,}
\providecommand{\dec}[2]{\ensuremath{\overset{#1}{\underset{#2}{\gtrless}}}}
\newcommand{\myvec}[1]{\ensuremath{\begin{pmatrix}#1\end{pmatrix}}}
\newcommand{\mydet}[1]{\ensuremath{\begin{vmatrix}#1\end{vmatrix}}}
\numberwithin{equation}{subsection}
\makeatletter
\@addtoreset{figure}{problem}
\makeatother
\let\StandardTheFigure\thefigure
\let\vec\mathbf
\renewcommand{\thefigure}{\theproblem}
\def\putbox#1#2#3{\makebox[0in][l]{\makebox[#1][l]{}\raisebox{\baselineskip}[0in][0in]{\raisebox{#2}[0in][0in]{#3}}}}
\def\rightbox#1{\makebox[0in][r]{#1}}
\def\centbox#1{\makebox[0in]{#1}}
\def\topbox#1{\raisebox{-\baselineskip}[0in][0in]{#1}}
\def\midbox#1{\raisebox{-0.5\baselineskip}[0in][0in]{#1}}
\vspace{3cm}
\title{Assignment No.1}
\author{Valli Devi Bolla MD 704}
\maketitle
\newpage
\bigskip
\renewcommand{\thefigure}{\theenumi}
\renewcommand{\thetable}{\theenumi}
Download all python codes from
\begin{lstlisting}
https://github.com/Vallidevibolla/bolla1.git
\end{lstlisting}
%
and latex-tikz codes from
%
\begin{lstlisting}
https://github.com/Vallidevibolla/bolla1.git
\end{lstlisting}
%
\section{Question No.13}


In Fig. {\triangle}ABD}\ is\ a\ right\ triangle,\ right-\ angled\ at\ $\vec{A}$ \ and\ $\vec{AC}$ $\vec{BD}$.\ Prove\ that\  $\vec{AB^2}{=}{BC.BD}$.


\section{Solution}
In {\triangle}ABD}\ , {{AB^2+AD^2}{=}{BD^2}}\------$(1)$

In {\triangle}ABC}\ , {{AC^2+BC^2}{=}{AB^2}}\------$(2)$

In {\triangle}ACD}\ , {{AC^2+CD^2}{=}{AD^2}}\------$(3)$

{Subtracting $(3)$ from $(2)$}
        {AB^2-AD^2 {=}BC^2-CD^2}\------$(4)$
        
\  Adding $1$ and $4$     
         {2AB^2 {=}{BC^2+BD^2-CD^2}}
         
         {2AB^2 {=}{(BC+CD)^2+BC^2-CD^2}}
          
Since {BD=BC+CD}
     {2AB^2{=}{2BC^2+2BC.CD}}

{2AB^2{=}{(BC+CD)2BC}

\outcome {AB^2{=}{BC.BD}}
        
        
         Hence it is proved that {AB^2{=}{BC.BD}}

\begin{tikzpicture}

 
    \node[rectangle,draw] (r) at (1,4) {};

\end{tikzpicture}

\begin{figure}[h!]
\includegraphics[width=\linewidth]{download.png}
  \caption{Right angled triangle}
  \label{Right angled triangle}
\end{figure}


\section{2.Question 9}

In Fig. {\triangle}ABC}\ is\ circumscribing\ a\ circle.\ Find\ the\ length\ of \ $\vec{BC}$.


\section{Solution}

Given {BR{=}$3cm$}
{AR{=}$4cm$}
{AC{=}$11cm$}


{BP{=}BR} 


{AQ{=}AR}       


{CP{=}CQ}

{(Because length of tangents to circle from external point will be equal)}

Therefore 
{AQ{=}4$4cm$}       {BP{=}$3cm$}

As {AC{=}$11cm$}
           
           
           {QC+AQ{=}$11cm$}
            {QC{=}$11$-{AQ}}
            
            
            
            {QC{=}$7cm$}            {PC{=}$7cm$}
            
            
 {BC{=}BP+PC
 
 
 
 {BC{=}3+7}   {BC{=}$10cm$}
 
 
 
 The length of BCis $10cm$
 
 

\begin{tikzpicture}

 
    \node[rectangle,draw] (r) at (1,4) {};

end{tikzpicture}

\begin{figure}[h!]
\includegraphics[width=\linewidth]{Tangent.png}
  \caption{tangent lines to circle radius 4 units.}
  \label{tangent lines to circle of radius 4 units.}
\end{figure}




    




\end{document}
